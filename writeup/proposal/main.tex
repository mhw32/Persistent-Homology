\documentclass[12pt]{article} 
\usepackage[margin=1in]{geometry} 
\usepackage{amsmath,amsthm,amssymb}
\usepackage{amsmath,amsfonts,amssymb,amsthm,epsfig,epstopdf,titling,url,array}
\usepackage{graphicx}
\begin{document}

\title{Comparative Analysis of the Large-Scale Structure of the Universe under Varying Assumptions}
\author{Mike Wu}

\maketitle
\section{Motivation}
The global structure of the Universe is thought to be composed of a distribution of matter and energy. By a large margin, dark energy is the most common element, thought to permeate all of space, contributing to as much as 68.3\% percent of the observable universe. Dark matter, a hypothetical form of matter that neither absorbs nor emits light, is thought to occupy a remaining 26.8\% of the observable universe, leaving baryonic matter, or ordinary matter such as the stars, planets, and humans, only 4.9\%. 

These definitions, while accepted by the majority of the cosmological community, are based heavily on assumptions. There is little evidence of dark matter existing as a particle, and researchers have not been able to create or consistently measure such a particle. Given that the only evidence for dark matter is observation of gravitational forces, there is a question of whether dark matter truly exists or if there are edge cases to our understanding of gravity.

Provided this uncertainty, one approach to gaining understanding of the true structure of the Universe is through cosmological simulations. Given the capacity to measure the observable universe, scientists have well-documented its current state. By changing the assumptions by which the simulations are built on, i.e. hydrodynamical forces, decoupling and cooling factors, cosmic wind speeds, one can find the set of initial conditions that best reflect the true Universe. However, in order to find the best initial conditions, one has to be able to compare different simulations. Simulations produce complex structures, and simple image analysis algorithms will not be able to distinguish finer details. To better make comparisons, an algorithm must exist that can locate a wide variety of objects in complex space.


\end{document}


